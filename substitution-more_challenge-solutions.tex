\documentclass[12pt, letterpaper]{article}

\usepackage[letterpaper,left=0.7in,right=0.7in,top=\dimexpr15mm+1.5\baselineskip,bottom=0.7in]{geometry}
\usepackage{amsmath}
\usepackage{pxfonts}
\usepackage[T1]{fontenc}

\begin{document}

$$ \int \frac{dt}{t\sqrt{\log{t}}} $$

Most of this problem set was "easy" (except the last one) if you made the right observation. This problem \emph{might} have seemed easier if I had written it as

\begin{equation*}
\int \left( \log{t} \right)^{-1/2} \frac{1}{t} \, dt
\end{equation*}

because you might have more easily seen that

\begin{equation*}
u = \log{t} \quad \Rightarrow \quad du = \frac{1}{t} \, dt.
\end{equation*}

Now this is as easy a problem as we might have:


\begin{equation*}
\begin{aligned}
\int \frac{dt}{t\sqrt{\log{t}}}
  &= \int \frac{du}{\sqrt{u}} \\[0.2in]
  &= \int u^{-1/2} \, du \\[0.2in]
  &= \frac{u^{1/2}}{\frac{1}{2}} + C \\[0.2in]
  &= 2\sqrt{u} + C \\[0.2in]
  &= 2\sqrt{\log{t}} + C \\[0.2in]
  &= \sqrt{4\log{t}} + C \\[0.2in]
  &= \sqrt{\log{\left( t^4 \right)}} + C
\end{aligned}
\end{equation*}

where I got saucy with the $\log$ at the end for fun. So we get

\begin{equation*}
\boxed
{
\int \frac{dt}{t\sqrt{\log{t}}}
  = \sqrt{\log{\left( t^4 \right)}} + C.
}
\end{equation*}



\newpage

$$ \int \sqrt{\frac{1 - t}{1 + t}} \, dt $$

This one isn't really terrible, you just have to kind of put your head down and persevere. You can rationalize the numerator or denominator, it doesn't really matter. I'll do the denominator:

\begin{equation*}
\begin{aligned}
\int \sqrt{\frac{1 - t}{1 + t}} \, dt
  &= \int \frac{\sqrt{1 - t}}{\sqrt{1 + t}}\frac{\sqrt{1 - t}}{\sqrt{1 - t}} \, dt \\[0.2in]
  &= \int \frac{1-t}{\sqrt{1 - t^2}} \, dt \\[0.2in]
  &= \int \frac{dt}{\sqrt{1 - t^2}} + \int \frac{-t \, dt}{\sqrt{1 - t^2}}
\end{aligned}
\end{equation*}

The second of these two integrals is an easy $u$-sub. Let $u = 1 - t^2$, then $\tfrac{1}{2}\, du = -t\, dt$, which gives

\begin{equation*}
\begin{aligned}
\int \frac{-t \, dt}{\sqrt{1 - t^2}}
  &= \tfrac{1}{2} \int \frac{du}{\sqrt{u}} \\[0.2in]
  &= \tfrac{1}{2} \frac{\sqrt{u}}{\frac{1}{2}} \\[0.2in]
  &= \sqrt{1 - t^2}.
\end{aligned}
\end{equation*}

For the first integral, we can use trig-sub, letting $t = \sin{\theta}$ so that $dt = \cos{\theta} \, d\theta$, which gives

\begin{equation*}
\begin{aligned}
\int \frac{dt}{\sqrt{1 - t^2}}
  &= \int \frac{\cos{\theta}}{\sqrt{1 - \sin^2{\theta}}} \, d\theta \\[0.2in]
  &= \int \frac{\cos{\theta}}{\cos{\theta}} \, d\theta \\[0.2in]
  &= \int \, d\theta \\[0.2in]
  &= \theta \\[0.2in]
  &= \arcsin{t}.
\end{aligned}
\end{equation*}

So we get

\begin{equation*}
\boxed
{
\int \sqrt{\frac{1 - t}{1 + t}} \, dt
  = \arcsin{t} + \sqrt{1 - t^2} + C.
}
\end{equation*}








\newpage

$$ \int \frac{e^{\sin{\psi}}}{\sec{\psi}} \, d\psi $$

This is another one that's "easy" if you spot the right substitution. Remembering (or looking up) that $\sec{\psi} = \frac{1}{\cos{\psi}}$, we get

\begin{equation*}
\begin{aligned}
\int \frac{e^{\sin{\psi}}}{\sec{\psi}} \, d\psi
  = \int \frac{e^{\sin{\psi}}}{\frac{1}{\cos{\psi}}} \, d\psi
  = \int e^{\sin{\psi}}\cos{\psi} \, d\psi
\end{aligned}
\end{equation*}

Now letting $u = \sin{\psi}$ we get $du = \cos{\psi} \, d\psi$, which gives

\begin{equation*}
\begin{aligned}
\int \frac{e^{\sin{\psi}}}{\sec{\psi}} \, d\psi
  &= \int e^{\sin{\psi}}\cos{\psi} \, d\psi \\[0.2in]
  &= \int e^{u}\, du \\[0.2in]
  &= e^u + C \\[0.2in]
  &= e^{\sin{\psi}} + C.
\end{aligned}
\end{equation*}

So we have

\begin{equation*}
\boxed
{
\int \frac{e^{\sin{\psi}}}{\sec{\psi}} \, d\psi
  = e^{\sin{\psi}} + C.
}
\end{equation*}





\newpage

$$ \int \frac{\tan^3{\psi}}{\cos^3{\psi}} \, d\psi $$

This could make a good test problem... There's nothing much to this one other than to rewrite $\frac{1}{\cos^3{\psi}} = \sec^3{\psi}$. Then we have odd powers of $\tan$ and $\sec$ and can use the strategy from section 7.2 to deal with this by letting $u = \sec{\psi}$ so that $du = \tan{\psi}\sec{\psi} \, d\psi$.

\begin{equation*}
\begin{aligned}
\int \frac{\tan^3{\psi}}{\cos^3{\psi}} \, d\psi
  &= \int \tan^3{\psi}\sec^3{\psi} \, d\psi \\[0.2in]
  &= \int \tan{\psi}\left( \sec^2{\psi} - 1 \right) \sec^3{\psi} \, d\psi \\[0.2in]
  &= \int \tan{\psi} \sec^5{\psi} \, d\psi - \int \tan{\psi}\sec^3{\psi} \, d\psi \\[0.2in]
  &= \int \sec^4{\psi} \tan{\psi} \sec{\psi} \, d\psi - \int \sec^2{\psi} \tan{\psi}\sec{\psi} \, d\psi \\[0.2in]
  &= \int u^4 \, du - \int u^2 \, du \\[0.2in]
  &= \tfrac{1}{5}u^5 - \tfrac{1}{3}u^3 + C \\[0.2in]
  &= \tfrac{1}{5}\sec^5{\psi} - \tfrac{1}{3}\sec^3{\psi} + C. \\[0.2in]
\end{aligned}
\end{equation*}

So we have

\begin{equation*}
\boxed
{
\int \frac{\tan^3{\psi}}{\cos^3{\psi}} \, d\psi
  = \tfrac{1}{5}\sec^5{\psi} - \tfrac{1}{3}\sec^3{\psi} + C. \\[0.2in]
}
\end{equation*}









\newpage

$$ \int \frac{\arctan{x}}{x^2} \, dx $$

This one is easiest to evaluate if we decide to "get rid of" the $\arctan{x}$ via substitution, although it's tedious. Since $\arctan$ is the inverse function of $\tan$, by definition we have $\arctan{\tan{\theta}} = \theta$. Therefore let $x = \tan{\theta}$ so that $dx = \sec^2{\theta}\, d\theta$. This gives

\begin{equation*}
\begin{aligned}
\int \frac{\arctan{x}}{x^2} \, dx
  &= \int \frac{\arctan{\left( \tan{\theta} \right) }}{\tan^2{\theta}} \sec^2{\theta} \, d\theta \\[0.2in]
  &= \int \frac{\theta}{\tan^2{\theta}} \left( 1 + \tan^2{\theta} \right) \, d\theta \\[0.2in]
  &= \int \theta \, d\theta + \int \theta \cot^2{\theta} \, d\theta
\end{aligned}
\end{equation*}

The right most integral is a classic IBP situation. Let $u = \theta$ so that $du = d\theta$. Then $dv = \cot^2{\theta} \, d\theta$ and we have

\begin{equation*}
\begin{aligned}
v
  &= \int \cot^2{\theta} \, d\theta \\[0.2in]
  &= \int \frac{\cos^2{\theta}}{\sin^2{\theta}} \, d\theta \\[0.2in]
  &= \int \frac{1 - \sin^2{\theta}}{\sin^2{\theta}} \, d\theta \\[0.2in]
  &= \int \csc^2{\theta} \, d\theta - \int \, d\theta \\[0.2in]
  &= -\cot{\theta} - \theta.
\end{aligned}
\end{equation*}

Plugging this in to the original integral for our IBP, we get

\begin{equation*}
\begin{aligned}
\int \frac{\arctan{x}}{x^2} \, dx
  &= \int \theta \, d\theta + \int \theta \cot^2{\theta} \, d\theta \\[0.2in]
  &= \int \theta \, d\theta - \theta \cot{\theta} - \theta^2 + \int \left( \cot{\theta} + \theta \right) \, d\theta \\[0.2in]
  &= 2\int \theta \, d\theta + \int \cot{\theta}\, d\theta - \theta \cot{\theta} - \theta^2
\end{aligned}
\end{equation*}

Let's to $\int \cot{\theta}\, d\theta$ by $u$-sub: Let $u = \sin{\theta}$, then $du = \cos{\theta} \, d\theta$. This gives

\begin{equation*}
\begin{aligned}
\int \cot{\theta}\, d\theta
  &= \int \frac{\cos{\theta}}{\sin{\theta}} \, d\theta \\[0.2in]
  &= \int \frac{du}{u} \\[0.2in]
  &= \log{\left| u \right|} \\[0.2in]
  &= \log{\left| \sin{\theta} \right|}.
\end{aligned}
\end{equation*}

Plugging this in above, we get

\begin{equation*}
\begin{aligned}
\int \frac{\arctan{x}}{x^2} \, dx
  &= 2\int \theta \, d\theta + \int \cot{\theta}\, d\theta - \theta \cot{\theta} - \theta^2 \\[0.2in]
  &= \theta^2 + \log{|\sin{\theta}|} - \theta \cot{\theta} - \theta^2 + C \\[0.2in]
  &= \log{|\sin{\theta}|} - \theta \cot{\theta} + C \\[0.2in]
  &= \log{\left| \frac{x}{\sqrt{x^2 + 1}} \right|} - \frac{\arctan{x}}{x} + C
\end{aligned}
\end{equation*}

where we used $x = \tan{\theta}$ to draw a triangle that gives $\sin{\theta} = \frac{x}{\sqrt{x^2 + 1}}$ and $\cot{\theta} = \frac{1}{x}$. Thus we have

\begin{equation*}
\boxed
{
\int \frac{\arctan{x}}{x^2} \, dx
  = \log{\left| \frac{x}{\sqrt{x^2 + 1}} \right|} - \frac{\arctan{x}}{x} + C
}
\end{equation*}







\end{document}































